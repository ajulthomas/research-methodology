\documentclass[a4paper,12pt]{article}
\usepackage{geometry}
\geometry{a4paper, margin=1in}
\usepackage{url}
\usepackage{enumitem}

\title{Assignment 1: Review and Critique of Research Paper \& Methodology}
\author{}
\date{}

\begin{document}
\maketitle
\begin{flushleft}

\textbf{Unit: Information Sciences Research Methodology PG (6797)}

\vspace{0.5cm}

\textbf{Article Title:} \\
Experimental Evaluation 

\vspace{0.3cm}

\textbf{Authors:} \\
Arunan Sivanathan

\vspace{0.3cm}

\textbf{Access:} \url{https://uclearn.canberra.edu.au/} \\
\textit{30-08-2024}

\vspace{0.5cm}

\section*{My Research Topic}
\textbf{Topic:} \\
Exploring and Mitigating Emerging Cybersecurity Threats in the Evolving Household IoT Devices

\vspace{0.3cm}

\textbf{Research Questions:}
\begin{itemize}[noitemsep]
    \item What are the emerging cybersecurity threats within Household IoT devices?
    \item What are the key vulnerabilities and attack vectors associated with household IoT devices and how do these vulnerabilities contribute to emerging threats?
    \item What technologies and strategies can be employed to mitigate the impact of emerging threats in household IoT ecosystems and how effective are these mitigation measures?
\end{itemize}

\vspace{0.3cm}

\textbf{Research Objectives:}
\begin{itemize}[noitemsep]
    \item To identify and categorize the types of emerging cybersecurity threats targeting household IoT devices.
    \item To analyze the key vulnerabilities and attack vectors associated with household IoT devices.
    \item To evaluate the effectiveness of different mitigation strategies in reducing the impact of identified emerging threats.
\end{itemize}

\vspace{0.3cm}

\textbf{Keywords:} \\
privacy risks.

\vspace{0.5cm}

\section*{How or Why Did You Choose This Paper?}
I chose the paper \cite{paper1} 

\vspace{0.5cm}

\section*{Title or Abstract of the Paper}
The title of the paper clearly describes

\vspace{0.5cm}

\section*{Research Problem}
The research focuses on evaluating 

\vspace{0.5cm}

\section*{Theories or Framework}
The authors employ a .

\vspace{0.5cm}

\section*{Methodology and Methods}
The paper \cite{paper1} has elements of both quantitative and qualitative approaches, 


\vspace{0.5cm}

\section*{Research Contributions}
This paper contributes to both theory and practice in the field of IoT security. 

\vspace{0.5cm}

\section*{Research Quality}
The experimental tests conducted are well-defined and systematically executed, ensuring that the results accurately reflect the security vulnerabilities of the tested IoT devices. 

\vspace{0.5cm}

\section*{Ethical Issues}
The paper does not explicitly detail the ethical considerations or issues addressed in the research paper. 

\vspace{0.5cm}

\section*{Deficiencies}
The paper exhibits multiple deficiencies and gaps.

\vspace{0.5cm}

\section*{Bibliography}
\begin{itemize}
    \item[{[1]}] Sivanathan, A., Loi, F., Gharakheili, H. H., \& Sivaraman, V. (2017). Experimental Evaluation of Cybersecurity Threats to the Smart-Home. In 2017 IEEE International Conference on Advanced Networks and Telecommunications Systems (ANTS).
    \item[{[2]}] ResearchGate (2017). ResearchGate. Available at: \url{https://www.researchgate.net/publication/325911350_Experimental_evaluation_of_cybersecurity_threats_to_the_smart-home/citations}. [Accessed August 2023].
    \item[{[3]}] Cisco Systems (2016). Visual networking Index (VNI). Available at: \url{http://www.cisco.com}.
\end{itemize}

\end{flushleft}
\end{document}
